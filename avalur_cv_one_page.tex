\documentclass[11pt]{article}

\usepackage{mathtext}         % если нужны русские буквы в формулах (не обязательно)
\usepackage[T2A]{fontenc}     % внутренняя T2A кодировка TeX
\usepackage[russian]{babel}   % включение переносов
\usepackage[utf8]{inputenc}
\usepackage[margin=0.5in]{geometry}
\pagestyle{empty} % нумерация выкл.
\addtolength{\textheight}{1.75in}
\usepackage{hyperref}
\usepackage{longtable}
\usepackage{color}
\usepackage{setspace}
\definecolor{gray}{rgb}{0.4,0.4,0.4}

\usepackage{fontawesome}	% иконки скайпа, телефона, ... http://ctan.sharelatex.com/tex-archive/fonts/fontawesome/doc/fontawesome.pdf

\setstretch{0.8}

\begin{document}

\noindent {\Huge{\textbf{Александр Авдюшенко}}}%

\vspace{0.5em}

\noindent Санкт-Петербург, Россия $\bullet$ \href{mailto:ovalur@gmail.com}{ovalur@gmail.com} \faMobile~+7 913 718 53 78
% \faSendO~ovalur
\faGithub~\href{https://github.com/avalur}{avalur}

\vspace{0.5em}
\hrule
\vspace{0.5em}

Руководитель проектов, преподаватель и аналитик с более чем 7-летним опытом работы в области машинного обучения и образовательных программ для ИТ-индустрии.

\vspace{0.5em}
\noindent {\textbf{Опыт работы}}
\begin{itemize}
	\item 2022 — н.в.:
	Руководитель ML-программ, \textbf{\href{https://yandex.ru}{Яндекс}}
	\begin{itemize}
		\item Организовал годовой курс машинного обучения на 350+ студентов в 2022-23 гг.
		\item Набрал 100+ талантливых студентов на три партнёрские магистратуры Яндекса
	\end{itemize}

	\item 2019 — н.в.: доцент и руководитель программы по анализу данных, \textbf{\href{https://spbu.ru}{Санкт-Петербургский государственный университет}, Факультет математики и компьютерных наук}

	\begin{itemize}
		\item Запустил образовательную программу бакалавриата \href{https://maad.compscicenter.ru}{<<Науки о данных>>} в 2019, сразу ставшую популярной среди лучших абитуриентов России
		\item Создал с нуля на грант министерства образования РФ \href{https://gsom.spbu.ru/all_news/event2021-02-04/}{Международный научно-методический центр СПбГУ} в 2020
		\item Разработал и читаю \href{https://github.com/spbu-math-cs/ml-course/}{годовой курс машинного обучения} по современным подходам в машинном обучении. Средняя студенческая оценка преподавания — 4,78 из 5,0
		\item Организую ежегодную \href{https://sochisirius.ru/obuchenie/nauka/smena1078/5204}{смену в Сириусе} по математике и компьютерным наукам
	\end{itemize}

	\item 2013 — н.в.: куратор академических программ, \textbf{\href{https://compscicenter.ru}{Яндекс, CS center/Школа анализа данных}}
	\begin{itemize}
		\item запустил филиал ШАД в Новосибирске в 2013
		\item в сотрудничестве с компанией Jetbrains расширил филиал ШАД до отделения \href{https://compscicenter.ru}{Computer science center} в 2017
		\item организовал \href{https://sochisirius.ru/obuchenie/graduates/smena240/1174}{<<Интенсивный проектный практикум для разработчиков от Сириуса, Яндекса и ФКН ВШЭ>>} в 2019
	\end{itemize}

	\item 2015 — 2018: аналитик-разработчик (data scientist), \textbf{\href{https://yandex.ru/}{Яндекс}}
	\begin{itemize}
		\item Дважды повысил точность организаций Справочника на 2\% и 3\% (главная цель команды)
		\item Автоматизировал и оптимизировал процессы актуализации данных в колл-центрах и в Толоке (краудсорс)
		\item В команде из 6 человек применяли машинное обучение, писали на Java, SQL, Python, немного С++
	\end{itemize}

	\item 2009 — 2015: научный сотрудник, \textbf{\href{http://www.ict.nsc.ru}{Институт вычислительных технологий СО РАН}}
	\begin{itemize}
		\item Разработал новую численную модель, учитывающую изменение во времени геометрии проточного тракта гидротурбины
		\item Ускорил расчеты имеющимся в лаборатории комплексом программ на языке Fortran в 16 раз (MPI по блокам), и ещё в 7 раз (OpenMP-потоки внутри блока)
		\item Защитил \href{https://github.com/avalur/dissertation/blob/master/to_print/autoref_Avdyushenko.pdf}{кандидатскую диссертацию <<Новые численные модели гидродинамики турбомашин>>}
	\end{itemize}
% 2008 — 2014 & {\href{http://sesc.nsu.ru}{Специализированный учебно-научный центр НГУ}, преподаватель математики}\\
% 2006 — 2017 & {организатор и член Жюри математических олимпиад: Всероссийская, Всесибирская, Турнир городов}\\
% 2010 — 2013 & {\href{https://www.sovenok.academy/}{мат. кружок Совёнок}, организатор и преподаватель}\\
\end{itemize}

\noindent {\textbf{Навыки}}
\begin{longtable} {p{0.5\textwidth}p{0.5\textwidth}}
\vspace{-2.0em}
\begin{itemize}
	\item Python (numpy, scikit-learn, matplotlib), SQL
	% \item Java (spring, hibernate)
	\item linux, git, ipython notebook
	\item MapReduce, MPI, OpenMP
\end{itemize}
&
\vspace{-2.0em}
\begin{itemize}
	\item machine learning (catboost, neural nets)
	% \item алгоритмы, структуры данных
	\item математика (анализ, численные методы)
	\item русский: родной, английский: продвинутый, французский: базовый
\end{itemize}
\\
\end{longtable}

\vspace{-2.5em}

\noindent {\textbf{Образование \vspace{-0.5em}}}
\begin{longtable} {p{0.3\textwidth}p{0.5\textwidth}l}
  магистр математики & \href{https://www.nsu.ru/}{Новосибирский государственный университет} & 2009 \vspace{0.5em} \\
	к.ф.-м.н & \href{http://www.ict.nsc.ru/ru/education/postgraduate}{аспирантура ИВТ СО РАН} & 2014 \vspace{0.5em}\\
  \href{https://academy.yandex.ru/dataschool/life}{выпускник} & \href{https://yandexdataschool.ru}{Школа анализа данных, Яндекс} & 2016 \vspace{0.5em}\\
\end{longtable}

\vspace{-1.0em}

\noindent {\textbf{Профессиональное развитие}}
\vspace{-0.5em}
\begin{longtable} {p{0.35\textwidth}p{0.25\textwidth}ll}
	Programming Languages, parts A, B, C &
	University of Washington & \href{https://coursera.org/share/3e187e640ed6df57b0a84ecb8a82ddab}{Coursera сертификат} & 2017 \vspace{0.5em}\\
	How to Win a Data Science Competition & HSE University & \href{https://coursera.org/share/0ad3f02a08a405800c29d5909caece90}{Coursera сертификат} & 2019   \vspace{0.5em}\\
\end{longtable}

\vspace{-1.0em}

% \noindent {\textbf{Вне работы}}

% Люблю проводить время с женой и двумя дочками.

\end{document}

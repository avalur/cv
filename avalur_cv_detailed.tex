\documentclass[11pt]{article}

\usepackage{mathtext}         % если нужны русские буквы в формулах (не обязательно)
\usepackage[T2A]{fontenc}     % внутренняя T2A кодировка TeX
\usepackage[russian]{babel}   % включение переносов
\usepackage[utf8]{inputenc}
\usepackage[margin=0.5in]{geometry}
\pagestyle{empty} % нумерация выкл.
\addtolength{\textheight}{1.75in}
\usepackage{hyperref}
\usepackage{longtable}
\usepackage{color}
\usepackage{setspace}
\definecolor{gray}{rgb}{0.4,0.4,0.4}

\usepackage{fontawesome}	% иконки скайпа, телефона, ... http://ctan.sharelatex.com/tex-archive/fonts/fontawesome/doc/fontawesome.pdf


% в полной версии много букв, поэтому поплотнее
\setstretch{0.8}

\begin{document}

\noindent {\Huge{\textbf{Александр Авдюшенко}}}

\vspace{0.5em}

\noindent Санкт-Петербург, Россия $\bullet$ \href{mailto:ovalur@gmail.com}{ovalur@gmail.com} \faMobile~+7 913 718 53 78 \faSendO~ovalur
% \faGithub

\vspace{0.5em}
\hrule
\vspace{0.5em}

\noindent {\textbf{Опыт работы}}
\begin{longtable} {l | p{0.85\textwidth}}
2019 — н.в. & {\textbf{\href{https://spbu.ru}{Санкт-Петербургский государственный университет}, Факультет математики и компьютерных наук, доцент и руководитель программы по анализу данных}} \\
&
\begin{itemize}
	\item с 2019 — руководитель новой образовательной программы бакалавриата \href{https://maad.compscicenter.ru}{<<Науки о данных>>} (до 2021 — <<Математика, алгоритмы и анализ данных>>), созданной в партнёрстве с Яндексом на факультете математики и компьютерных наук СПбГУ.

	Программа сочетает фундаментальное математическое образование с изучением современного программирования и машинного обучения. Студентам предлагаются темы научных исследований, связанные с разработками ведущих компаний в ИТ. Программа сразу стала популярной среди мотивированных абитуриентов: в 2019 году конкурс был около 25 человек на место. Практически все бюджетные места заполняются победителями и призёрами олимпиад по математике и информатике, включая заключительный этап Всероссийской олимпиады школьников и международную олимпиаду по математике (IMO).

	\item 2019-2020 — создал с нуля на грант министерства образования РФ \href{https://gsom.spbu.ru/all_news/event2021-02-04/}{Международный научно-методический центр СПбГУ}

	В 2019-2020 годах руководил командой факультета математики и компьютерных наук в совместном с институтом «Высшая школа менеджмента» СПбГУ проекте Международного научно-методического центра СПбГУ (МНМЦ).

	Концепция МНМЦ заключалась в создании учебно-методических материалов и реализации передовых программ повышения квалификации и переподготовки преподавателей, активном включении лучших российских и зарубежных экспертов на всех этапах работы и, как следствие, формированию, поддержке и развитию профессионального сообщества преподавателей университетов.

	Всего к реализации проекта было привлечено 123 человека, среди которых 61 кандидат наук или PhD (50\%) и 24 доктора наук (20\%). Команда проекта не только успешно справилась с возникшими в условиях пандемии COVID-19 вызовами, быстро перестроившись и выполнив запланированное, но и обернула их на пользу проекту, осуществив интенсивное обучение научно-педагогических работников и аспирантов российских образовательных организаций со всей страны.

	За 2020 год в рамках МНМЦ было проведено две программы повышения квалификации, две программы стажировки, одна программа профессиональной переподготовки и одна школа по математике и компьютерным наукам для преподавателей университетов Российской Федерации. Всего на мероприятиях МНМЦ прошли обучение более 1500 человек из более чем 50 городов и 60 вузов России. Мероприятия получили теплый отклик сообщества, а выпускники центра с удовольствием используют полученные знания и методические материалы в своей работе.

  \item 2020-2022 — соруководитель и преподаватель ежегодных \href{https://sochisirius.ru/obuchenie/nauka/smena1078/5204}{<<Январских научных школ по математике и программированию>>} в Сириусе

	\item 2021 — все бюджетные места программы <<Науки о данных>> заполняются олимпиадниками-всеросами, включая двух победителей международной олимпиады по математике

	\item 2021-2022 — читаю \href{https://github.com/spbu-math-cs/ml-course/}{годовой курс машинного обучения} по современным подходам. Средняя студенческая оценка преподавания — 4,78 из 5,0.
\end{itemize}
\\
2013 — н.в. & {\textbf{\href{https://compscicenter.ru}{Яндекс, CS center/Школа анализа данных}, куратор академических программ}} \\
&
\begin{itemize}
	\item 2013 — запустил филиал \href{https://yandexdataschool.ru}{ШАД} в Новосибирске. C 2016 по 2018 гг. в сумме 16 выпускников
	\item 2017 — в сотрудничестве с компанией JetBrains расширил филиал ШАДа до отделения \href{https://compscicenter.ru}{Computer science center}
	\item 2019 — преподаватель и руководитель на \href{https://sochisirius.ru/obuchenie/graduates/smena240/1174}{<<Интенсивном проектном практикуме для разработчиков от Сириуса, Яндекса и ФКН ВШЭ>>}
\end{itemize} \\

\end{longtable}
\newpage
\begin{longtable} {l | p{0.85\textwidth}}

2015 — 2018 & {\textbf{\href{https://yandex.ru/}{Яндекс}, аналитик-разработчик (data scientist)}} \\
&
\begin{itemize}
	\item реализовал ежедневный расчет поатрибутной (наличие организации, её название, адрес, время работы...) метрики точности базы организаций Яндекса
	\item автоматизировал и оптимизировал процессы актуализации данных в колл-центрах и в \href{https://toloka.yandex.ru}{Яндекс.Толоке}. В частности, на 20\% улучшил эффективность актуализации, применив catboost для предсказания вероятности закрытия организации
	\item дважды повысил точность организаций Справочника на 2\% и 3\%, выделив из потерь наиболее крупные проблемы и исправив их: аналитическая поддержка процесса обхода компаний с недоступными телефонами; повышение точности времени работы организаций их приоритетной актуализацией операторами колл-центров
\end{itemize}
\\

2009 — 2015 & {\textbf{\href{http://www.ict.nsc.ru}{Институт вычислительных технологий СО РАН}, аспирант, научный сотрудник}}\\
&
\begin{itemize}
	\item обобщил на нестационарную геометрию метод решения трехмерных уравнений Рейнольдса движения несжимаемой жидкости
	\item ускорил расчеты в 16 раз, распараллелив алгоритм по блокам расчетной области с использованием MPI для процессов с распределенной памятью (улучшение имеющегося в лаборатории комплекса программ на языке Fortran, $\sim$30 тыс. строк кода)
	\item ускорил ещё в 7 раз, распараллелив алгоритм в одном блоке с использованием OpenMP-потоков с общей памятью
	\item опубликовал 19 научных работ (4 статьи, 13 тезисов, 2 патента), защитив в итоге \href{https://github.com/avalur/dissertation/blob/master/to_print/autoref_Avdyushenko.pdf}{кандидатскую диссертацию <<Новые численные модели гидродинамики турбомашин>>}
\end{itemize}
\\

2008 — 2014 & {\href{http://sesc.nsu.ru}{Специализированный учебно-научный центр НГУ}, преподаватель математики}\\
2006 — 2017 & {организатор и член Жюри математических олимпиад: Всероссийская, Всесибирская, Турнир городов}\\
2010 — 2013 & {\href{https://www.sovenok.academy/}{мат. кружок Совёнок}, организатор и преподаватель}\\

\end{longtable}

\noindent {\textbf{Навыки}}
\begin{longtable} {p{0.5\textwidth}p{0.5\textwidth}}
\vspace{-2.2em}
\begin{itemize}
	\item Python (numpy, scikit-learn, matplotlib)
	\item Java (spring, hibernate, immutables)
	\item linux, git, ipython notebook
	\item SQL, MapReduce, MPI, OpenMP
\end{itemize}
&
\vspace{-2.2em}
\begin{itemize}
	\item machine learning (catboost, neural nets)
	\item алгоритмы, структуры данных
	\item математика (анализ, численные методы)
	\item русский: родной, английский: intermediate, французский: débutant
\end{itemize}
\\
\end{longtable}

\vspace{-0.5em}

\noindent {\textbf{Образование \vspace{-0.5em}}}
\begin{longtable} {cp{0.35\textwidth}p{0.38\textwidth}l}
2013 — 2016	& \href{https://yandexdataschool.ru}{Школа анализа данных, Яндекс} & программирование, машинное обучение & выпускник \vspace{0.5em}\\
2009 — 2014	& \href{http://www.ict.nsc.ru/ru/education/postgraduate}{аспирантура ИВТ СО РАН} & мат. моделирование, численные методы & к.ф.-м.н \vspace{0.5em}\\
2003 — 2009	& \href{https://www.nsu.ru/n/mathematics-mechanics-department/}{НГУ, механико-математический факультет} & математика & магистр \vspace{0.5em}\\
\end{longtable}

\noindent {\textbf{Онлайн-курсы \vspace{-0.5em}}}
\begin{longtable} {cp{0.35\textwidth}p{0.38\textwidth}l}
2017 & Coursera, University of Washington & Programming Languages, parts A, B, C & 	сертификат \vspace{0.5em}\\
2019 & Coursera, HSE University & How to Win a Data Science Competition: Learn from Top Kagglers & сертификат \vspace{0.5em}\\
\end{longtable}

\noindent {\textbf{Вне работы}}

\vspace{0.5em} Люблю проводить время с женой и дочкой, стараюсь по возможности регулярно заниматься спортом (плавание, скалолазание), есть в \href{https://www.instagram.com/ovalur/}{Insragram}

\end{document}

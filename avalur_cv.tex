\documentclass[11pt]{article}

\usepackage{mathtext}         % если нужны русские буквы в формулах (не обязательно)
\usepackage[T2A]{fontenc}     % внутренняя T2A кодировка TeX
\usepackage[russian]{babel}   % включение переносов
\usepackage[utf8]{inputenc}
\usepackage[margin=0.5in]{geometry}
\pagestyle{empty} % нумерация выкл.
\addtolength{\textheight}{1.75in}
\usepackage{hyperref}
\usepackage{longtable}
\usepackage{color}
\usepackage{setspace}
\definecolor{gray}{rgb}{0.4,0.4,0.4}

\usepackage{fontawesome}	% иконки скайпа, телефона, ... http://ctan.sharelatex.com/tex-archive/fonts/fontawesome/doc/fontawesome.pdf

\newif\ifdetailed

% для полной версии резюме — detailedtrue, для сокращенной — detailedfalse
\detailedtrue
%\detailedfalse

\ifdetailed
% в полной версии много букв, поэтому поплотнее
\setstretch{0.8}
\fi

\begin{document}

\noindent {\Huge{\textbf{   Александр Авдюшенко   }}}%

\vspace{0.5em}

\noindent Новосибирск, Россия $\bullet$ \href{mailto:ovalur@gmail.com}{ovalur@gmail.com} \faMobile~+7 913 718 53 78 \faSendO~ovalur
% \faGithub

\vspace{0.5em}
\hrule
\vspace{1.0em}

\noindent {\textbf{Опыт работы}}
\begin{longtable} {l | p{0.85\textwidth}}

2013 — н.в. & {\textbf{\href{https://compscicenter.ru}{Яндекс, CS center/Школа анализа данных}, куратор академических программ}} \\
\ifdetailed
& \vspace{-1em}
\begin{itemize}
	\item в 2013 запустил филиал \href{https://yandexdataschool.ru}{ШАД} в Новосибирске. C 2016 по 2018 гг. в сумме 16 выпускников
	\item в 2017 г в сотрудничестве с компанией JetBrains расширил филиал ШАДа до \href{https://compscicenter.ru}{Computer science center}
\end{itemize}
\\
\else
& \vspace{-1em} {В 2013 запустил филиал ШАД в Новосибирске (с 2016 по 2018 гг. в сумме 16 выпускников). В 2017 в сотрудничестве с компанией Jetbrains расширил филиал ШАДа до \href{https://compscicenter.ru}{Computer science center}.} \\
\fi


2015 — 2018 & {\textbf{\href{https://yandex.ru/sprav/main}{Яндекс.Справочник}, аналитик}} \\
\ifdetailed
& \vspace{-1em}
\begin{itemize}
	\item реализовал ежедневный расчет поатрибутной (публикуемость, название, адрес, время работы...) метрики точности базы организаций Яндекса
	\item автоматизировал и оптимизировал процессы актуализации данных в колл-центрах и в \href{https://toloka.yandex.ru}{Яндекс.Толоке}. В частности, на 20\% улучшил эффективность актуализации, применив catboost для предсказания вероятности закрытия организации
	\item дважды повысил точность организаций Справочника на 2\% и 3\%, выделив из потерь наиболее крупные проблемы и исправив их. Во-первых, исправив компании с недоступными телефонами. Во-вторых, повысив точность времени работы организаций их приоритетной актуализацией
\end{itemize}
\\
\else
& \vspace{-1em} {Дважды повысил точность организаций Справочника на 2\% и 3\%. Автоматизировал и оптимизировал процессы актуализации данных в колл-центрах и в Толоке. Работа в группе из 6 человек, разработка на Java, SQL, Python, немного С++; catboost.} \\
\fi


2009 — 2015 & {\textbf{\href{http://www.ict.nsc.ru}{Институт вычислительных технологий СО РАН}, аспирант, научный сотрудник}}\\
\ifdetailed
& \vspace{-1em}
\begin{itemize}
	\item обобщил на нестационарную геометрию метод решения трехмерных уравнений Рейнольдса движения несжимаемой жидкости
	\item ускорил расчеты в 16 раз, распараллелив алгоритм по блокам расчетной области с использованием MPI для процессов с распределенной памятью (улучшение имеющегося в лаборатории комплекса программ на языке Fortran, $\sim$30 тыс. строк кода)
	\item ускорил ещё в 7 раз, распараллелив алгоритм в одном блоке с использованием OpenMP-потоков с общей памятью
	\item опубликовал 19 научных работ (4 статьи, 13 тезисов, 2 патента), защитив в итоге кандидатскую <<Новые численные модели гидродинамики турбомашин>>
\end{itemize}
\\
\else
& \vspace{-1em} {Разработал новую численную модель, учитывающую изменение геометрии проточного тракта во времени. Ускорил расчеты имеющимся в лаборатории комплексом программ на языке Fortran в 16 раз (MPI по блокам), и ещё в 7 раз (OpenMP-потоки внутри блока). Защитил кандидатскую диссертацию.} \\
\fi

\ifdetailed
2008 — 2014 & {\href{http://sesc.nsu.ru}{СУНЦ НГУ}, преподаватель математики}\\
2006 — 2017 & {организатор и член Жюри математических олимпиад: Всероссийская, Всесибирская, Турнир городов}\\
2010 — 2013 & {\href{https://sites.google.com/site/sovenoknsk/}{мат. кружок Совёнок}, организатор и преподаватель}\\
\fi

\end{longtable}

\ifdetailed
\noindent {\textbf{Навыки}}
\begin{longtable} {p{0.5\textwidth}p{0.5\textwidth}}
\vspace{-2em}
\begin{itemize}
	\item Python (numpy, scikit-learn, matplotlib)
	\item Java (spring, hibernate, immutables)
	\item linux, git, ipython notebook
	\item SQL, MapReduce, MPI, OpenMP
\end{itemize}
& 
\vspace{-2em}
\begin{itemize}
	\item machine learning (catboost, SVM, RF)
	\item алгоритмы, структуры данных
	\item математика (матстат, анализ, алгебра)
	\item русский: родной, английский: intermediate, французский: débutant
\end{itemize}
\\
\end{longtable}
\fi

\vspace{-2em}

\ifdetailed
\noindent {\textbf{Образование \vspace{-0.5em}}}
\begin{longtable} {cp{0.35\textwidth}p{0.38\textwidth}l} 
2013 — 2016	& \href{https://yandexdataschool.ru/about/graduates/yearbook/2016}{Школа анализа данных, Яндекс} & программирование, машинное обучение & выпускник \vspace{0.5em}\\
2009 — 2014	& \href{http://www.ict.nsc.ru/ru/education/postgraduate}{аспирантура ИВТ СО РАН} & мат. моделирование, численные методы & к.ф.-м.н \vspace{0.5em}\\
2003 — 2009	& \href{}{НГУ, механико-математический факультет} & математика & магистр \vspace{0.5em}\\
\end{longtable}

\noindent {\textbf{Курсы \vspace{-0.5em}}}
\begin{longtable} {cp{0.35\textwidth}p{0.38\textwidth}l} 
2017	& Coursera, University of Washington & Programming Languages, parts A, B, C & 	сертификат \vspace{0.5em}\\
\end{longtable}
\fi

\end{document}             % End of document.
